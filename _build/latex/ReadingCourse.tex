%% Generated by Sphinx.
\def\sphinxdocclass{jupyterBook}
\documentclass[letterpaper,10pt,english]{jupyterBook}
\ifdefined\pdfpxdimen
   \let\sphinxpxdimen\pdfpxdimen\else\newdimen\sphinxpxdimen
\fi \sphinxpxdimen=.75bp\relax
\ifdefined\pdfimageresolution
    \pdfimageresolution= \numexpr \dimexpr1in\relax/\sphinxpxdimen\relax
\fi
%% let collapsible pdf bookmarks panel have high depth per default
\PassOptionsToPackage{bookmarksdepth=5}{hyperref}
%% turn off hyperref patch of \index as sphinx.xdy xindy module takes care of
%% suitable \hyperpage mark-up, working around hyperref-xindy incompatibility
\PassOptionsToPackage{hyperindex=false}{hyperref}
%% memoir class requires extra handling
\makeatletter\@ifclassloaded{memoir}
{\ifdefined\memhyperindexfalse\memhyperindexfalse\fi}{}\makeatother

\PassOptionsToPackage{booktabs}{sphinx}
\PassOptionsToPackage{colorrows}{sphinx}

\PassOptionsToPackage{warn}{textcomp}

\catcode`^^^^00a0\active\protected\def^^^^00a0{\leavevmode\nobreak\ }
\usepackage{cmap}
\usepackage{fontspec}
\defaultfontfeatures[\rmfamily,\sffamily,\ttfamily]{}
\usepackage{amsmath,amssymb,amstext}
\usepackage{polyglossia}
\setmainlanguage{english}



\setmainfont{FreeSerif}[
  Extension      = .otf,
  UprightFont    = *,
  ItalicFont     = *Italic,
  BoldFont       = *Bold,
  BoldItalicFont = *BoldItalic
]
\setsansfont{FreeSans}[
  Extension      = .otf,
  UprightFont    = *,
  ItalicFont     = *Oblique,
  BoldFont       = *Bold,
  BoldItalicFont = *BoldOblique,
]
\setmonofont{FreeMono}[
  Extension      = .otf,
  UprightFont    = *,
  ItalicFont     = *Oblique,
  BoldFont       = *Bold,
  BoldItalicFont = *BoldOblique,
]



\usepackage[Bjarne]{fncychap}
\usepackage[,numfigreset=1,mathnumfig]{sphinx}

\fvset{fontsize=\small}
\usepackage{geometry}


% Include hyperref last.
\usepackage{hyperref}
% Fix anchor placement for figures with captions.
\usepackage{hypcap}% it must be loaded after hyperref.
% Set up styles of URL: it should be placed after hyperref.
\urlstyle{same}


\usepackage{sphinxmessages}



        % Start of preamble defined in sphinx-jupyterbook-latex %
         \usepackage[Latin,Greek]{ucharclasses}
        \usepackage{unicode-math}
        % fixing title of the toc
        \addto\captionsenglish{\renewcommand{\contentsname}{Contents}}
        \hypersetup{
            pdfencoding=auto,
            psdextra
        }
        % End of preamble defined in sphinx-jupyterbook-latex %
        

\title{Reading course}
\date{Aug 29, 2025}
\release{}
\author{Jean Braun}
\newcommand{\sphinxlogo}{\vbox{}}
\renewcommand{\releasename}{}
\makeindex
\begin{document}

\pagestyle{empty}
\sphinxmaketitle
\pagestyle{plain}
\sphinxtableofcontents
\pagestyle{normal}
\phantomsection\label{\detokenize{intro::doc}}


\sphinxAtStartPar
This reading course addresses the question of what controls the height of mountain belts. Is it the strength of the \DUrole{xref,std,std-term}{lithosphere} compared to the weight of the topography, the efficiency of erosion (and thus a function of climate) or the rate at which plates move at the Earth’s surface.

\sphinxAtStartPar
To illustrate this point, let’s look at a picture taken from space where we can appreciate how smooth the Earth’s surface is. The range of surface topography is under 10,000 m compared to the Earth’s radius of 6371 km. A ratio of approximately 1/600 making the Earth’s relief about the size of the imperfection in an orange peal compared to its size.

\begin{figure}[htbp]
\centering
\capstart

\noindent\sphinxincludegraphics[height=150\sphinxpxdimen]{{fromspace}.png}
\caption{Earth’s surface from space}\label{\detokenize{intro:smooth-earth}}\end{figure}
\subsubsection*{What is the stress caused by 1 km of surface topography?}

\sphinxAtStartPar
The stress caused by a a topography \(\Delta h\) is equal to \(\sigma=\rho g \Delta h\) where \(\rho\) is rock density and \(g\) the gravitational acceleration. Thus, assuming a value of 9.81 for \(g\) and that surface rocks have a density of approximately 2400 \(kg/m^3\), a 1000 m high topography causes a stress equal to:
\(\sigma = 2400\times 9.81 \times 1000 = 23.544\) MPa.

\sphinxAtStartPar
By comparison, Iapetus, one of Saturn’s satellites with a radius of only 734 km, displays a 13km\sphinxhyphen{}high equatorial ridge, yielding a relief/radius ratio of approximaelty 1/50. Is it due to the reduced gravity and thus weight of topography or to the absence of an atmosphere and thus of erosion processes that is responsible for this much taller relative relief?

\begin{figure}[htbp]
\centering
\capstart

\noindent\sphinxincludegraphics[height=150\sphinxpxdimen]{{iapetus}.png}
\caption{Iapetus and its equatorial bulge}\label{\detokenize{intro:iapetus}}\end{figure}

\sphinxAtStartPar
This question can be asked at the planetary scale but the answer is also highly relevant to understand the variability in mountain height at the Earth’s surface. The Earth’s topography is characterized by relatively flat and low\sphinxhyphen{}elevation continental interiors with narrow high\sphinxhyphen{}elevation mountain belts around their bondaries where plates have collided in the recent geological past. Examples of such orogenic belts include the European Alps, the Pyrenees, the Himalayas, the Cordillera or the Southern Alps of New Zealand. But there exists also extensive continental areas called plateaus that are characterized by high elevation and low relief. Examples of plateaus include the Tibetan plateau, the Altiplano but also the south African plateau.

\begin{figure}[htbp]
\centering
\capstart

\noindent\sphinxincludegraphics[height=300\sphinxpxdimen]{{worldtopo}.png}
\caption{Earth’s surface topography}\label{\detokenize{intro:worldtopo}}\end{figure}

\sphinxAtStartPar
We like to classify topography in two categories: the topography that is related to variations in crustal thickness through the principle of \DUrole{xref,std,std-term}{isostasy} and the topography that is not correlated to variations in crustal thickness and is thus considered to be caused not by tectonic plate interactions (collision or extension) but by flow in the underlying convective mantle. This later type of topography is usual termed \DUrole{xref,std,std-term}{dynamic topography}, in contrast to the former called isostatic or tectonic topography.

\begin{figure}[htbp]
\centering
\capstart

\noindent\sphinxincludegraphics[height=400\sphinxpxdimen]{{dynamictopo}.png}
\caption{Difference between dynamic topography (left) caused by convective flow in the Earth’s mantle and tectonic topography (right) caused by variations in crustal thcikness during collision or extension of tectonic plates}\label{\detokenize{intro:dynamictopo}}\end{figure}

\sphinxAtStartPar
In this course we will use a relatively small number of articles that each proposes a different hypothesis for the control of orogenic topography. It is important that participants read these articles and prepare a set of notes where they will not only summarize the main findings or conclusions but also the hypothesis that were made and the tools that were used.

\begin{sphinxuseclass}{sd-container-fluid}
\begin{sphinxuseclass}{sd-sphinx-override}
\begin{sphinxuseclass}{sd-mb-4}
\begin{sphinxuseclass}{sd-row}
\begin{sphinxuseclass}{sd-g-4}
\begin{sphinxuseclass}{sd-g-xs-4}
\begin{sphinxuseclass}{sd-g-sm-4}
\begin{sphinxuseclass}{sd-g-md-4}
\begin{sphinxuseclass}{sd-g-lg-4}
\begin{sphinxuseclass}{sd-col}
\begin{sphinxuseclass}{sd-d-flex-row}
\begin{sphinxuseclass}{sd-col-6}
\begin{sphinxuseclass}{sd-col-xs-6}
\begin{sphinxuseclass}{sd-col-sm-6}
\begin{sphinxuseclass}{sd-col-md-6}
\begin{sphinxuseclass}{sd-col-lg-6}
\begin{sphinxuseclass}{sd-card}
\begin{sphinxuseclass}{sd-sphinx-override}
\begin{sphinxuseclass}{sd-w-100}
\begin{sphinxuseclass}{sd-shadow-sm}
\begin{sphinxuseclass}{sd-card-body}
\begin{sphinxuseclass}{sd-card-title}
\begin{sphinxuseclass}{sd-font-weight-bold}Hypothesis 1
\end{sphinxuseclass}
\end{sphinxuseclass}
\sphinxAtStartPar
Strength and rate

\noindent{\hspace*{\fill}\sphinxincludegraphics[width=200\sphinxpxdimen]{{england-mckenzie}.png}\hspace*{\fill}}

\sphinxAtStartPar
\sphinxhref{https://doi.org/10.1111/j.1365-246X.1982.tb04969.x}{England and McKenzie, 1982}

\end{sphinxuseclass}
\end{sphinxuseclass}
\end{sphinxuseclass}
\end{sphinxuseclass}
\end{sphinxuseclass}
\end{sphinxuseclass}
\end{sphinxuseclass}
\end{sphinxuseclass}
\end{sphinxuseclass}
\end{sphinxuseclass}
\end{sphinxuseclass}
\end{sphinxuseclass}
\begin{sphinxuseclass}{sd-col}
\begin{sphinxuseclass}{sd-d-flex-row}
\begin{sphinxuseclass}{sd-col-6}
\begin{sphinxuseclass}{sd-col-xs-6}
\begin{sphinxuseclass}{sd-col-sm-6}
\begin{sphinxuseclass}{sd-col-md-6}
\begin{sphinxuseclass}{sd-col-lg-6}
\begin{sphinxuseclass}{sd-card}
\begin{sphinxuseclass}{sd-sphinx-override}
\begin{sphinxuseclass}{sd-w-100}
\begin{sphinxuseclass}{sd-shadow-sm}
\begin{sphinxuseclass}{sd-card-body}
\begin{sphinxuseclass}{sd-card-title}
\begin{sphinxuseclass}{sd-font-weight-bold}Hypothesis 2
\end{sphinxuseclass}
\end{sphinxuseclass}
\sphinxAtStartPar
Erosion and rate

\noindent{\hspace*{\fill}\sphinxincludegraphics[width=200\sphinxpxdimen]{{whipple-tucker}.png}\hspace*{\fill}}

\sphinxAtStartPar
\sphinxhref{https://doi.org/10.1029/1999JB900120}{Whipple and Tucker, 1999}

\end{sphinxuseclass}
\end{sphinxuseclass}
\end{sphinxuseclass}
\end{sphinxuseclass}
\end{sphinxuseclass}
\end{sphinxuseclass}
\end{sphinxuseclass}
\end{sphinxuseclass}
\end{sphinxuseclass}
\end{sphinxuseclass}
\end{sphinxuseclass}
\end{sphinxuseclass}
\begin{sphinxuseclass}{sd-col}
\begin{sphinxuseclass}{sd-d-flex-row}
\begin{sphinxuseclass}{sd-col-6}
\begin{sphinxuseclass}{sd-col-xs-6}
\begin{sphinxuseclass}{sd-col-sm-6}
\begin{sphinxuseclass}{sd-col-md-6}
\begin{sphinxuseclass}{sd-col-lg-6}
\begin{sphinxuseclass}{sd-card}
\begin{sphinxuseclass}{sd-sphinx-override}
\begin{sphinxuseclass}{sd-w-100}
\begin{sphinxuseclass}{sd-shadow-sm}
\begin{sphinxuseclass}{sd-card-body}
\begin{sphinxuseclass}{sd-card-title}
\begin{sphinxuseclass}{sd-font-weight-bold}Hypothesis 3
\end{sphinxuseclass}
\end{sphinxuseclass}
\sphinxAtStartPar
Strength, erosion and rate

\noindent{\hspace*{\fill}\sphinxincludegraphics[width=200\sphinxpxdimen]{{wolfetal}.png}\hspace*{\fill}}

\sphinxAtStartPar
\sphinxhref{https://doi.org/10.1038/s41586-022-04700-6}{Wolf et al, 2002}

\end{sphinxuseclass}
\end{sphinxuseclass}
\end{sphinxuseclass}
\end{sphinxuseclass}
\end{sphinxuseclass}
\end{sphinxuseclass}
\end{sphinxuseclass}
\end{sphinxuseclass}
\end{sphinxuseclass}
\end{sphinxuseclass}
\end{sphinxuseclass}
\end{sphinxuseclass}
\begin{sphinxuseclass}{sd-col}
\begin{sphinxuseclass}{sd-d-flex-row}
\begin{sphinxuseclass}{sd-col-6}
\begin{sphinxuseclass}{sd-col-xs-6}
\begin{sphinxuseclass}{sd-col-sm-6}
\begin{sphinxuseclass}{sd-col-md-6}
\begin{sphinxuseclass}{sd-col-lg-6}
\begin{sphinxuseclass}{sd-card}
\begin{sphinxuseclass}{sd-sphinx-override}
\begin{sphinxuseclass}{sd-w-100}
\begin{sphinxuseclass}{sd-shadow-sm}
\begin{sphinxuseclass}{sd-card-body}
\begin{sphinxuseclass}{sd-card-title}
\begin{sphinxuseclass}{sd-font-weight-bold}Hypothesis 4
\end{sphinxuseclass}
\end{sphinxuseclass}
\sphinxAtStartPar
Glacial buzzsaw

\noindent{\hspace*{\fill}\sphinxincludegraphics[width=200\sphinxpxdimen]{{egholmetal2009}.png}\hspace*{\fill}}

\sphinxAtStartPar
\sphinxhref{https://doi.org/10.1038/nature08263}{Egholm et al, 2009}

\end{sphinxuseclass}
\end{sphinxuseclass}
\end{sphinxuseclass}
\end{sphinxuseclass}
\end{sphinxuseclass}
\end{sphinxuseclass}
\end{sphinxuseclass}
\end{sphinxuseclass}
\end{sphinxuseclass}
\end{sphinxuseclass}
\end{sphinxuseclass}
\end{sphinxuseclass}
\end{sphinxuseclass}
\end{sphinxuseclass}
\end{sphinxuseclass}
\end{sphinxuseclass}
\end{sphinxuseclass}
\end{sphinxuseclass}
\end{sphinxuseclass}
\end{sphinxuseclass}
\end{sphinxuseclass}
\sphinxAtStartPar
In the context of this course, I have selected four papers that use numerical modeling as the main tool to support or develop their hypothesis. Furthermore, each paper quantifies their hypothesis by introduicing and assessing the value of a \DUrole{xref,std,std-term}{dimensionless number}.

\sphinxAtStartPar
I will end this introduction by highlighting the main objectives of the reading course:
\begin{enumerate}
\sphinxsetlistlabels{\arabic}{enumi}{enumii}{}{.}%
\item {} 
\sphinxAtStartPar
to demonstrate the usefulness of numerical models in addressing important questions

\item {} 
\sphinxAtStartPar
to understand what dimensionless numbers are and how they can be used to understand system behaviour

\item {} 
\sphinxAtStartPar
to improve your research skills, and in particular:
\begin{itemize}
\item {} 
\sphinxAtStartPar
how to read a scientific paper

\item {} 
\sphinxAtStartPar
how to take part in a scientific debate

\item {} 
\sphinxAtStartPar
how to present scientific results

\end{itemize}

\end{enumerate}

\sphinxstepscope


\chapter{Some basic concepts}
\label{\detokenize{concepts:some-basic-concepts}}\label{\detokenize{concepts::doc}}
\sphinxAtStartPar
This course relates to the use of numerical modeling as a tool to address basic questions about the Earth’s behaviour and, in particular in our case, about how tectonic, climate and rheology potentially all contribute to controlling the height of mountain topography on Earth. Numerical models use numerical methods that are used to solve partial differential equations. We will not study these methods but focus here on other tools that are commonly used to understand the behaviour of PDEs as a funciton of model parameter and bounday condition values, namely a dimensional analysis, which leads to the definition of dimensionless numbers.

\sphinxAtStartPar
For this, we will need to introduce a couple of partial differential equations, namely the Stokes equation and the stream power law, as well as other basic physical principles, such as the concept of isostasy.

\sphinxstepscope


\section{Principle of isostasy}
\label{\detokenize{isostasy:principle-of-isostasy}}\label{\detokenize{isostasy::doc}}
\sphinxAtStartPar
The principle of \DUrole{xref,std,std-term}{isostasy} is Archimede’s principle applied to the Earth’s \DUrole{xref,std,std-term}{lithosphere}/\DUrole{xref,std,std-term}{asthenosphere} system. It can be simply stated as \sphinxstylestrong{the weight of any lithospheric column must be the same}. It relies on the existence of a so\sphinxhyphen{}called compensation depth, which corresponds to a level in the Earth’s interior that has very low viscosity (the \DUrole{xref,std,std-term}{asthenosphere}) and thus behaves, on geological time scales, as an inviscid fluid. Such a fluid cannot sustain any horizontal stress and therefore the weight of any column of material above it must be uniform. This compensation depth is usually taken as the base of the lithosphere, which is commonly regarded as the depth where the geotherm is closest to the solidus and thus considered as the least viscous layer in the Earth’s upper regions.

\begin{sphinxadmonition}{warning}{Warning:}
\sphinxAtStartPar
Even though the viscosity of the asthenosphere is much lower than that of the overlying lithosphere, it remains finite. This means, for example, that the flow of asthenosphere caused by loading by an ice sheet will take several tens of thousands of years to bring the system into isostatic equilibrium. Although very large on human time scales, this time is very large compared to the tectonic time scales of the order of millions of years. This is why one can consider, when considering tectonic processes, that isostatic adjustment is instantaneous.
\end{sphinxadmonition}

\sphinxAtStartPar
From the definition of the principle of isostasy, one can see that it is most useful when used to compare the topography between two lithospheric columns, for example, the difference in height between a continental interior and a mountain belt.

\sphinxAtStartPar
In mathematical form, the principle of isostasy can be written as:
\begin{equation*}
\begin{split}\Delta \sigma_{zz}=g\int_{z_0}^L\Delta\rho\ dz=0\end{split}
\end{equation*}
\sphinxAtStartPar
where \(\Delta\rho\) is the difference in density between any two columns of lithosphere (as a function of \(z\)), \(L\) is the compensation depth, \(z_0\) is the height of the surface topography and \(g\) is the gravitational acceleration.

\begin{sphinxadmonition}{note}{Note:}
\sphinxAtStartPar
The principle of isostasy is a direct consequence of the quasi\sphinxhyphen{}static form of Newton’s second law that the sum of all forces at any point must be equal to zero.
\end{sphinxadmonition}

\sphinxAtStartPar
Note that, under the quasi\sphinxhyphen{}static approximation, Newton’s second loaw also implies that the sum of all torques (or force moments) at any point must be balances; this implies that, even if two lithospheric columns are in isostatic balance, there may exist a horizontal stress exerted by one column on the other:
\begin{equation*}
\begin{split}\bar\tau_{xx}=\frac{g}{L}\int_{z_0}^L\Delta\rho\ z\ dz\ne 0\end{split}
\end{equation*}
\sphinxAtStartPar
One also states that the two columns have different potential energy. This implies that unless an external force acts on the system, the thickened crust will naturally deform and thin, until its potential energy becomes identical to that of the reference column. This may lead, for example, to orogenic collapse once the tectonic force that drives continantal collision stops acting.

\begin{sphinxadmonition}{note}{Note:}
\sphinxAtStartPar
The principle of isostasy neglects the strength of the lithosphere and the potential it has for sustaining a differential horizontal stress. To address this limitation, one commonly assumes that the lithospehre behaves as a thin elastic plate that is able to sustain some of the stress imposed by additional topography (caused by crustal thickening for example) by flexing. This leads to the concept of \DUrole{xref,std,std-term}{flexural isostasy}.
\end{sphinxadmonition}

\sphinxstepscope


\subsection{A simple exercise on isostasy}
\label{\detokenize{exercise-isostasy:a-simple-exercise-on-isostasy}}\label{\detokenize{exercise-isostasy::doc}}
\sphinxAtStartPar
A simple exercise to compute the height difference between a reference continental interior and a mountain.
\begin{quote}

\sphinxAtStartPar
What is the elevation, \(e\), of a mountain resulting from thickening of a “normal”, \(h\) = 35 km thick crust and \(L\) = 125 km thick lithosphere, column by a factor \(\beta\)? Assume crustal, mantle and asthenospheric densities of \(\rho_c\) = 2800, \(\rho_m\) = 3200 and \(\rho_a\) = 3150 kg/m\(^3\), respectively.
\end{quote}

\begin{sphinxuseclass}{cell}\begin{sphinxVerbatimInput}

\begin{sphinxuseclass}{cell_input}
\begin{sphinxVerbatim}[commandchars=\\\{\}]
\PYG{k+kn}{import}\PYG{+w}{ }\PYG{n+nn}{numpy}\PYG{+w}{ }\PYG{k}{as}\PYG{+w}{ }\PYG{n+nn}{np}
\PYG{k+kn}{import}\PYG{+w}{ }\PYG{n+nn}{matplotlib}\PYG{n+nn}{.}\PYG{n+nn}{pyplot}\PYG{+w}{ }\PYG{k}{as}\PYG{+w}{ }\PYG{n+nn}{plt}
\end{sphinxVerbatim}

\end{sphinxuseclass}\end{sphinxVerbatimInput}

\end{sphinxuseclass}
\begin{sphinxuseclass}{cell}\begin{sphinxVerbatimInput}

\begin{sphinxuseclass}{cell_input}
\begin{sphinxVerbatim}[commandchars=\\\{\}]
\PYG{n}{rhoc} \PYG{o}{=} \PYG{l+m+mi}{2800}
\PYG{n}{rhom} \PYG{o}{=} \PYG{l+m+mi}{3200}
\PYG{n}{rhoa} \PYG{o}{=} \PYG{l+m+mi}{3150}
\PYG{n}{L} \PYG{o}{=} \PYG{l+m+mi}{125}
\PYG{n}{h} \PYG{o}{=} \PYG{l+m+mi}{35}
\end{sphinxVerbatim}

\end{sphinxuseclass}\end{sphinxVerbatimInput}

\end{sphinxuseclass}
\sphinxAtStartPar
\sphinxstyleemphasis{Solution}:

\begin{figure}[htbp]
\centering
\capstart

\noindent\sphinxincludegraphics[height=250\sphinxpxdimen]{{isostasy-sketch}.png}
\caption{Two lithospheric columns: a reference column with crustal thickness \(h\) and lithospheric thickness \(L\) of respective density \(\rho_c\) and \(\rho_m\) and a second column beneath a mountain belt with a crust and lithosphere thickened by a factor \(\beta\).}\label{\detokenize{exercise-isostasy:isostasy-sketch}}\end{figure}

\sphinxAtStartPar
Equating the weights of the two columns leads to:
\begin{equation*}
\begin{split}\rho_chg + \rho_m(L-h)g +\rho_a(\beta L-e-L)g = \rho_c\beta hg+\rho_m\beta(L-h)g\end{split}
\end{equation*}
\sphinxAtStartPar
which leads to the following expression for the elevation, \(e\) of the mountain:
\begin{equation*}
\begin{split}e = (\beta-1)\frac{(\rho_m-\rho_c)h-(\rho_m-\rho_a)L}{\rho_a} = (\beta-1)\times 2.46\end{split}
\end{equation*}
\begin{sphinxuseclass}{cell}\begin{sphinxVerbatimInput}

\begin{sphinxuseclass}{cell_input}
\begin{sphinxVerbatim}[commandchars=\\\{\}]
\PYG{n}{beta} \PYG{o}{=} \PYG{n}{np}\PYG{o}{.}\PYG{n}{linspace}\PYG{p}{(}\PYG{l+m+mi}{1}\PYG{p}{,}\PYG{l+m+mi}{4}\PYG{p}{,}\PYG{l+m+mi}{101}\PYG{p}{)}
\PYG{n}{e} \PYG{o}{=} \PYG{p}{(}\PYG{n}{beta}\PYG{o}{\PYGZhy{}}\PYG{l+m+mi}{1}\PYG{p}{)}\PYG{o}{*}\PYG{p}{(}\PYG{p}{(}\PYG{n}{rhom}\PYG{o}{\PYGZhy{}}\PYG{n}{rhoc}\PYG{p}{)}\PYG{o}{*}\PYG{n}{h} \PYG{o}{\PYGZhy{}} \PYG{p}{(}\PYG{n}{rhom}\PYG{o}{\PYGZhy{}}\PYG{n}{rhoa}\PYG{p}{)}\PYG{o}{*}\PYG{n}{L}\PYG{p}{)}\PYG{o}{/}\PYG{n}{rhoa}    
\end{sphinxVerbatim}

\end{sphinxuseclass}\end{sphinxVerbatimInput}

\end{sphinxuseclass}
\begin{sphinxuseclass}{cell}\begin{sphinxVerbatimInput}

\begin{sphinxuseclass}{cell_input}
\begin{sphinxVerbatim}[commandchars=\\\{\}]
\PYG{n}{fig}\PYG{p}{,} \PYG{n}{ax} \PYG{o}{=} \PYG{n}{plt}\PYG{o}{.}\PYG{n}{subplots}\PYG{p}{(}\PYG{n}{figsize}\PYG{o}{=}\PYG{p}{(}\PYG{l+m+mi}{7}\PYG{p}{,}\PYG{l+m+mi}{4}\PYG{p}{)}\PYG{p}{)}
\PYG{n}{ax}\PYG{o}{.}\PYG{n}{plot}\PYG{p}{(}\PYG{n}{beta}\PYG{p}{,}\PYG{n}{e}\PYG{p}{)}
\PYG{n}{ax}\PYG{o}{.}\PYG{n}{set\PYGZus{}xlabel}\PYG{p}{(}\PYG{l+s+sa}{r}\PYG{l+s+s1}{\PYGZsq{}}\PYG{l+s+s1}{\PYGZdl{}}\PYG{l+s+s1}{\PYGZbs{}}\PYG{l+s+s1}{beta\PYGZdl{}}\PYG{l+s+s1}{\PYGZsq{}}\PYG{p}{)}
\PYG{n}{ax}\PYG{o}{.}\PYG{n}{set\PYGZus{}ylabel}\PYG{p}{(}\PYG{l+s+sa}{r}\PYG{l+s+s1}{\PYGZsq{}}\PYG{l+s+s1}{\PYGZdl{}e\PYGZdl{} (km)}\PYG{l+s+s1}{\PYGZsq{}}\PYG{p}{)}\PYG{p}{;}
\end{sphinxVerbatim}

\end{sphinxuseclass}\end{sphinxVerbatimInput}
\begin{sphinxVerbatimOutput}

\begin{sphinxuseclass}{cell_output}
\noindent\sphinxincludegraphics{{546e66343b161fba4ada2b3945a6308cc6b58c4a82afea01978b4b5b32cbcc75}.png}

\end{sphinxuseclass}\end{sphinxVerbatimOutput}

\end{sphinxuseclass}
\sphinxstepscope


\section{Thin sheet model}
\label{\detokenize{thinsheet:thin-sheet-model}}\label{\detokenize{thinsheet::doc}}
\sphinxAtStartPar
The thin sheet model is based on a simplified version of the Stokes equation that governs the flow of highly viscous (inertia\sphinxhyphen{}free) fluid.

\sphinxAtStartPar
It is commonly used to model tectonic processes at the scale of the plates. Its main limitations are that it can only  accurately represents deformation that takes place at a horizontal scale that is larger than the plate thickness and that it uses depth averaged properties for the plate.

\sphinxAtStartPar
To derive the equation governing the deformation of a thin viscous plate, we proceed in two steps. First we derive the general form of Stokes equation that governs the flow of a viscous, incompressible and inertia free fluid. Second, we introduce the simplifications that will lead to the thin sheet approximation.

\sphinxAtStartPar
This procedure is inspired (or is a summarized version of) the derivation given in England and McKenzie (1992).

\sphinxstepscope


\subsection{Stokes flow}
\label{\detokenize{stokes:stokes-flow}}\label{\detokenize{stokes::doc}}
\sphinxAtStartPar
Stokes eqaution is a simplified form of Navier\sphinxhyphen{}Stokes equation that assumes that one can neglect inertial forces (and thus turbulence). It takes the following form:
\begin{equation}\label{equation:stokes:stokes-force}
\begin{split}\frac{\partial\tau_{ij}}{\partial x_j}=\frac{\partial p}{\partial x_i}-\rho gz_i\end{split}
\end{equation}
\sphinxAtStartPar
where \(\tau_{ij}\) are the compnents of the stress tensor, \(p\) the pressure, \(\rho\) the density, \(g\) the gravitational acceleration and \(z_i\) the components of a vector pointing in the direction of \(g\). \(x_i\) represents the spatial coordinates.

\begin{sphinxadmonition}{note}{Note:}
\sphinxAtStartPar
A Stoke fluid is, by definition, devoid of inertia which implies that if the driving force stops the fluid motion stops instantaneously. Honey behaves like a Stoke fluid: if you stir a honey jar with a spoon, once you remove the spoon, the honey stops moving instantly. This behaviour is also observed in the Earth on geological time scales: it explains why hot spot chains (like the \sphinxhref{https://www.usgs.gov/media/images/hawaiian-emperor-seamount-chain}{Hawaiian\sphinxhyphen{}Emperor seamont chain}) display very abrupt strike changes that correspond to “instantaneous” .
\end{sphinxadmonition}

\sphinxAtStartPar
This equation represents the balance between three forces that must exist at any point within the fluid:
\begin{enumerate}
\sphinxsetlistlabels{\arabic}{enumi}{enumii}{}{.}%
\item {} 
\sphinxAtStartPar
the first term (on the left\sphinxhyphen{}hand side of the equation) represents the viscous forces

\item {} 
\sphinxAtStartPar
the second term (on the right\sphinxhyphen{}hand side of the equation) represents the forces arising from pressure gradient inside the fluid;

\item {} 
\sphinxAtStartPar
the third term (last on the right) represents the gravitational forces.

\end{enumerate}

\sphinxAtStartPar
Note that this equation does not contain any time dependency. In other words, the flow is static unless the boundary conditions or one of the parameters (the distribution of density for example) varies with time.

\begin{sphinxadmonition}{note}{Note:}
\sphinxAtStartPar
Stokes flow have interesting properties. One of them is that they are perfectly reversible. This can be appreciated on the \sphinxhref{https://www.youtube.com/watch?v=rA6T83FKuE8}{following video}
\end{sphinxadmonition}

\sphinxAtStartPar
If we wish to use Stokes equation to represent the motion and deformation of plates and flow in the underlying mantle, we must also assume that the flow is incompressible. Mathematically, this is equivalent to assuming that, at every point within the fluid, the divergence of the velocity vector is nil. In other words:
\begin{equation}\label{equation:stokes:incompressibility}
\begin{split}\frac{\partial u_i}{\partial x_i}=0\end{split}
\end{equation}
\sphinxAtStartPar
where \(u_i\) are the components of the velocity vector. Note that in this equation as in equation \eqref{equation:stokes:stokes-force}, we have used \sphinxhref{https://en.wikipedia.org/wiki/Einstein\_notation}{Einstein’s notation} which assumes that if an index appears more than once in a single term and is not otherwise defined, it implies summation.

\sphinxAtStartPar
We see that the first equation \eqref{equation:stokes:stokes-force} refers to stresses (or forces), whereas the second \eqref{equation:stokes:incompressibility} referes to velocities. To combine the two equations, one needs to introduce a rheological law, i.e., a rule that defines the relationship between deformation (velocity) and stress (force). Here we assume a non\sphinxhyphen{}linear viscous behaviour of the form:
\begin{equation}\label{equation:stokes:rheology}
\begin{split}\epsilon_{ij} = B^{-n}T^{n-1}\sigma_{ij}\end{split}
\end{equation}
\sphinxAtStartPar
where \(\epsilon_{ij}\) is the strain rate tensor, defined from the spatial derivatives of the velocity vector:
\begin{equation}\label{equation:stokes:strain-rate}
\begin{split}\epsilon_{ij}=\frac{1}{2}\Bigl(\frac{\partial u_i}{\partial x_j}+\frac{\partial u_j}{\partial x_i}\Bigr)\end{split}
\end{equation}
\sphinxAtStartPar
\(B\) is a constant and \(n\) an exponent that represents the non\sphinxhyphen{}linearity of the rheological law (the rheology is linear is \(n=1\)). Using values of \(n>1\) implies that the material is strain softening, i.e., its viscosity drops as the strain rate increases. \(T\) is the second invariant of the deviatoric part of the stress tensor. The deviatoric part means that the pressure has been substracted from the diagonal elements of the stress tensor. The second invariant is a special combination of the tensors components that remains invariant in any system of reference (the pressure is the first invariant of the stress tensor). More information on the invariants of a tensor can be found \sphinxhref{https://en.wikipedia.org/wiki/Invariants\_of\_tensors}{here}.

\sphinxstepscope


\subsection{Thin sheet approximation}
\label{\detokenize{approx:thin-sheet-approximation}}\label{\detokenize{approx::doc}}
\sphinxAtStartPar
The thin sheet approximation implies:
\begin{enumerate}
\sphinxsetlistlabels{\arabic}{enumi}{enumii}{}{.}%
\item {} 
\sphinxAtStartPar
that vertical columns in the sheet cannot be sheared

\item {} 
\sphinxAtStartPar
that all variables and properties can be approximated by their vertically averaged values

\end{enumerate}

\sphinxAtStartPar
This can be expressed mathematically as:
\begin{equation*}
\begin{split}\epsilon_{13}=\epsilon_{31}=\epsilon_{23}=\epsilon_{32}=0\end{split}
\end{equation*}
\sphinxAtStartPar
where the 3rd index corresponds to the vertical component and by integrating Stokes equation along the vertical dimension.

\sphinxAtStartPar
This yields:
\begin{equation*}
\begin{split}B\dot E^{1/n-1}\frac{\dot\epsilon_{ij}}{\partial x_j}+B(1/n-1)\frac{\partial \dot E}{\partial x_j}\dot E^{1/n-2}\dot\epsilon_{ij}=\frac{\partial\bar p}{\partial x_i}\end{split}
\end{equation*}
\sphinxAtStartPar
using the incompressibility (equation\eqref{equation:stokes:incompressibility}) condition that can be expressed in terms of the strin rate components:
\begin{equation*}
\begin{split}\epsilon_{33}=-(\epsilon_{11}+\epsilon_{22})\end{split}
\end{equation*}
\sphinxAtStartPar
\(\bar p\) is the vertically averaged pressure:
\begin{equation*}
\begin{split}\bar p=\frac{1}{L}\int_0^L p\ dz=\frac{g\rho_ch^2}{2L}(1-\rho_c/\rho_m)+\frac{g\rho_mL}{2}\end{split}
\end{equation*}
\sphinxAtStartPar
Combining the last two equations yields:
\begin{equation*}
\begin{split}\frac{1}{2}\frac{\partial}{\partial x_j}(\frac{\partial u_j}{\partial x_i}+\frac{\partial u_i}{\partial x_j})=\frac{g\rho_c h}{BL}(1-\rho_c/\rho_m)\dot E^{1-1/n}\frac{\partial h}{\partial x_i}+(1-1/n)\dot E^{-1}\frac{\partial\dot E}{\partial x_j}\dot\epsilon_{ij}\end{split}
\end{equation*}
\sphinxAtStartPar
where the indices \(i\) and \(j\) now only vary between 1 and 2.







\renewcommand{\indexname}{Index}
\printindex
\end{document}